%
\section{Literature Survey}\label{sec:literature survey}
%
This is the last chapter of this thesis \cite{tur38}.
%
\subsection{Evolution of Vulnerability Analysis}
Vulnerability analysis is an important operation to perform in all domains which also includes the software product. One single vulnerability can cause a catastrophe in an active system and which can be an open gateway to the attackers to exploit the system. A good vulnerability assessment should define, identify and categorize the issues in the component. 
Neumann and Parker say that the new vulnerabilities of IT systems are evolved from the attacks that happened in the past by using long-known techniques[21]. Therefore building a strong and secure application is merely an impossible task or will be more expensive. Earlier the vulnerability assessment happens manually  in all organizations where this is a huge and overwhelming work for the IT security teams. 
\paragraph{}
All the newly identified vulnerabilities should be stored in excel or a CSV file for later use. When compared to the no. of vulnerabilities today, the 90’s and late 2000 had very less number of vulnerabilities. The early vulnerability scanning software just gives a simple report of found vulnerabilities in the system and later this report has to be given to the IT security team to analyse the possible threats of the vulnerability. Then the report is sent to higher authority for a review and approval. This is such a manual process which has been used earlier to detect vulnerabilities in an active system[22]. Manual scanning and repair strategies would soon become impractical as the number of vulnerabilities grew in following years and the necessity of vulnerability management became more apparent to companies. Now the future of vulnerability analysis is focussing on fully automated assessment.
\subsubsection{Benefits}
Lutz Lowis describes that attackers may usually repeat their exploits by reusing a susceptible service[20]. Despite many old and new threats there are few advantages to performing vulnerability analysis operations. Here are some advantages that can be achieved through a vulnerability assessment[20][22][23]:
\begin{description}
	\item [$\bullet$]A vulnerability analysis can identify all the possible vulnerabilities in the system where this can be identified by both organization and the attacker(hacker) if they use the same software for scanning. The organization has a higher advantage by fixing this issue before the attackers initiate. 
	
	\item [$\bullet$]If the IT security person conducts a regular scanning of the system, the scanning can give the level of risk available in the system. So this helps to figure out the health of the overall system.
	
	\item [$\bullet$]Vulnerability assessment will save money and time because if an organization or individual fails to complete the vulnerability analysis procedure, there is a greater possibility that an attacker will be able to exploit the system, resulting in the system having to be rebuilt.
	
	\item [$\bullet$]The previous evaluation reports will be useful in improving the present system in the future.
\end{description}
\subsubsection{Vulnerability Attacks}
A vulnerability is an attribute of a software component that has a potential of exploiting or damaging an active system. Most of the vulnerabilities have a high tendency of causing damage to security policy by internal or external persons(hacker or insider). There is a past event which happened in 2017 where a big data breach occurred in Equifax. This data breach caused more than 100 million user data to be leaked[4]. In this vast area of IT there are still new and unknown vulnerabilities emerging everyday.
\paragraph{}
Ryohei Koizumi and Ryoichi Sasaki have found that whenever a vulnerability assessment operation takes place the IT security team should always use the latest scanning tool because sometimes the old softwares is effective against some small virus threat but it does not exploit the vulnerability[24]. There are some vulnerability attacks which have to be focused more because these type attacks are listed as high in risk level. Here are some major attack types that are focused by few researchers[25]:
\begin{description}
	\item [$\bullet$ Configuration-based:] This type of vulnerability occurs when there is a misconfiguration in the system or running any unwanted services in the background. The configuration-based vulnerability has a major weak point where the hackers can easily get into the organization network and try to find the active system by any form of misconfiguration. This type of vulnerability is based on weak management protocols, weak permissions and weak encryption.
	
	\item [$\bullet$ Security patches:] A security patch is an important update for all active systems because a role of a security patches is to fix or remove the flaw or issue which is found from the vulnerability report. The software patches play a vital role in rectifying and fixing the components for both commercial software and open-source software components. Generally the security patches will be released every month for example Microsoft sends security patches to its windows operating system every month. A system which fails to update their security patches will lead to major security breaches. As everyone knows, the best example of security patches failures is ransomware which is called wannacry[26].
	
	\item [$\bullet$ Zero-Day Vulnerability:] This is the most difficult vulnerability to rectify and fix the issue, this is because the vulnerability is new and unknown to the organization. The attackers will take advantage of this loophole and will cause a major security risk. The term “Zero-day” is used because the vendor was unaware of the threat which affected the software and the vendors had “0” days worked on the security patch or fixin the vulnerability.
	
	\item [$\bullet$ Faulty Open-Source Package:] This the vulnerability where hackers were using it for several years. Generally the hackers used to inject a credential sniffer inside a very useful common library or package. This kind of attack happens when the vendor doesn’t update the open-source packages regularly. This threat will stay in the package for several days or until it is found by the vendor.
\end{description}

\subsubsection{Types of Vulnerability Analysis}
The vulnerability analysis is also called vulnerability assessment where the main intended purpose of this is to keep the organization safe from the digital threats. This is a methodology which is used to find the IT application and the infrastructure. It also involves intense scanning by the security expert or team of the organization. There are few types of vulnerability analysis which are used exclusively for some part of IT:

{\bf Network-based Analysis:} A network-based analysis is a mechanism to identify the network defects and issues in the network. They mostly scan and analyze the network endpoint and device network for security issues. The failure of vulnerability analysis will give an opportunity to the hackers to take advantage of the network issue. The organization will invest more time to improve their existing framework which is used for network vulnerability analyses. In 2008 Hai L Vu etl, developed a vulnerability analysis framework for scanning network vulnerabilities and along with that they have also proposed a scalable algorithm. Both framework and algorithm are used to evaluate the network vulnerabilities without generating a full-scale graph[27]. This figure gives the results of each network component used in the network by using the framework. The results come with a brief description about the effects of the listed vulnerability. These information about the vulnerabilities have been taken with the help of vulnerability databases[27].

{\bf Host-based Analysis:} Sometimes a vulnerability can be found in the vendor's resources and with this vulnerability there are high chances where even an insider can be an attacker for the system. The attackers mostly cause damage by making an improper configuration setting in the host.This vulnerability assessment takes place in servers, workstations or other network hosts. The host-based analysis will give a detailed insight of configuration settings in the network, patches and update history. The insight which is gathered from the analysis will also give us the potential damage caused by the attackers or intruders. Anil Sharma et. al. created a software tool “Ferret” by using perl language that simply identifies the vulnerabilities present in the host[28]. This software tool helps the system administrator to identify the vulnerabilities and take action based on the threat. The host vulnerability is checked by using a different plug-in module and the end output will also mention which plug-in module is used for the assessment.

{\bf Database Analysis:} Misconfigurations occur often in databases and Big Data systems. Database vulnerability analysis is mainly used for identifying the available risk in the databases. The most common risks are missing patches, weak passwords and default vendor accounts[29]. Sartaj Singh described the importance of inherent dangers of the database like how data theft is happening in the internet era and he also mentioned that the existing encryption methods are not fool proof for the high end professionals[31]. A vulnerability attack can exploit file permission, database configuration files and also have potential to steal sensitive information like credit card details, personal details, etc. There is still research going to secure the database more effectively and efficiently. In 2008 Ghassan Jabbour and Daniel A. Menasce presented a framework that provides a self protection to the database from unauthorized or intensional security parameter changes and also they proposed that this framework can be implemented in an Oracle 10g Release 2 database[32].

{\bf Application Analysis:} Vulnerabilities are frequently identified in third-party apps that are built and managed. The vulnerability assessment is important to an organization’s security team because they have to identify the vulnerability in the application before it exploits the system. This process is used to identify vulnerabilities which are misconfiguration in applications, outdated software packages and weak authentication. Sultan S. Alqahtani[33] has researched a modeling approach that improves traceability and trust in software products by linking the security knowledge with the software artifacts. He also introduced a scanner called  Semantic Global Problem Scanner(SE-GPS) which is created by integrating the modeling approach and with the modeling approach the tool can now link the NVD[13] security database to the maven build repository. 
\paragraph{}
The application vulnerability analysis is an important security process to be considered by the organization. The process's main goal is to identify the vulnerability and report it to the security authority to mitigate it before the vulnerability exploits the system. Most application vulnerability analysis tools use the proper guidelines to make a good scanning tool. Here is the main guideline that a good vulnerability scanner tool should use[34]:
\begin{description}
	\item [$\bullet$ Setup:] The setup should begin with a proper documentation about the application, with perfect security permissions and configuration tools.
	
	\item [$\bullet$ Test Execution:] Run the scanner tool which can be an existing tool or a self created one. The scanner should identify all the software packages and its dependencies used inside the software product.
	
	\item [$\bullet$ Vulnerability Analysis:] Once the execution is finished, the extracted software packages and its dependencies should be analysed by using any existing vulnerability databases like NVD, ISS(Internet Security Systems) , etc. each software package is searched in the database by using its name and version.
	
	\item [$\bullet$ Reporting \& Remediation:] After the previous process, the analysis will give a proper report of each software package. The report includes the risk level of each package which is categorized as LOW , MEDIUM and  HIGH. Sometimes the report will also provide a brief description of the threat. The remediation action is taken by the security department of the organization. Mostly the remediation will have two possibilities which is to change the version of the package or to find an alternate software package.
\end{description}
\subsubsection{Vulnerability Databases}

\subsection{Data Scraping}
\subsubsection{Types of Data Scraping}
\subsubsection{Challenges}
\subsubsection{Data Scraping Methods}

\subsection{String Distance Metrics}
\subsubsection{Hamming Distance}
\subsubsection{Levenshtein distance}
\subsubsection{Damerau-Levenshtein distance}
\subsubsection{Jaro Distance}