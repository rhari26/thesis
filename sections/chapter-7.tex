%
\section{Conclusion \& Future Work}\label{sec:conclusion}

\subsection{Conclusion}
In this thesis, we have focused on automating the open-source software discovery by developing a client-side scanner and finding a suitable vulnerability database to find the \acs{OSS} component’s vulnerability. Software security is one of the most significant quality assurance measures for software. It has been shown that software security is one of the most critical and essential parts to focus on before the application gets delivered to the end user[4]. This proposed scanner is developed to find all the OSS components used in a software project, and this solution will help the developers to find the vulnerabilities present in the \acs{OSS} component. The scanner can scan the following application framework projects: Django, Laravel, Ruby on Rails, .Net core, Gradle and Maven.

This implementation has the advantage that the \acs{OSS} component analyzer module can be reused for integrating with another vulnerability database. Also, it can be modified by adding a new application framework project for scanning. The end report provided by the system will not take or suggest any decisions on behalf of the user. The report will give the vulnerability findings of each component where the developer can decide whether to keep the same version or move to a different version of the \acs{OSS} component. This thesis implementation will be helpful for the first level of security check before delivering the application to the end-user.

\subsection{Future Work}
Our work shows a direction for further development and research in the domain of open-source software security. We have developed a scanner that extracts the \acs{OSS} components from software projects with the help of dependency managers. There are numerous
possibilities to improve or customize this scanner according to the application scenario.

The extracted \acs{OSS} component from the software projects can be used for creating a license clearing tool. The license clearing tool is another crucial process in an organization to maintain transparency towards software usage. If the license is adequately maintained, then the organization can avoid the unwanted cost e.g. copyleft issues \cite{copyleft}.

Apart from automating the extraction of \acs{OSS} components and delivering the vulnerability information of each \acs{OSS} component, we can also develop an extra feature like recommending the nearest version with less vulnerability of the same component. In this case, the user can have an idea of the nearest version of the component.

There could be many other ways to improve the scanner, other than the ones pointed out by us. However, we believe that the scanner so far has yielded satisfactory results, as is evident from the evaluation. We hope to see more improvements in this scanner, or in general, in the field of open-source software security.
%