%
\section{Conclusion \& Future Work}\label{sec:conclusion}

\subsection{Conclusion}
In this thesis we have focused on automating the open-source software discovery by developing a client side scanner and finding a suitable vulnerability database to find the \acs{OSS} component’s vulnerability. Software security is one of the most significant quality assurance measures for software. It has been shown that software security is one of the most critical and important parts to focus on before the application gets delivered to the end user[4]. This proposed scanner is developed to find all the OSS components used in a software project and this solution will help the developers to find the vulnerabilities present in the \acs{OSS} component. The scanner can scan the following application framework projects: Django, Laravel, Ruby on Rails, .Net core, Gradle and Maven.

This implementation have an advantage that the \acs{OSS} component analyzer module can be reused for integrating with another vulnerability database and also it can be modified by adding new application framework project for scanning. The end report which is provided by system will not take or suggest any decisions behalf of the user. The report will give you the vulnerability findings of each component where the developer can decide whether to keep the same version or move to different version of the \acs{OSS} component. This thesis implementation will be helpful fo the first level of security check before delivering the application to end user.

\subsection{Future Work}
My work shows a direction for further development and research in the domain of open-source software security. I have developed a scanner that extracts the \acs{OSS} components from software projects with the help dependency managers. There are numerous
possibilities to improve or customize this scanner according to the application scenario.

The extracted \acs{OSS} component from the software projects can be used for creating a license clearing tool. The license clearing tool is another important process in an organization so that the organization can maintain transparency towards the usage of a software.If the license is properly maintained then the organization can avoid unwanted cost.

Apart from automating the extraction of \acs{OSS} components and delivering the vulnerability information of each \acs{OSS} component, we can also develop an extra feature like recommending the nearest version with less vulnerability of the same component. In this case the user can have an idea of nearest version of the component.

There could be many other ways in which the scanner can be improved, other than the ones pointed out by me. However, I believe that the scanner so far has yielded satisfactory results as is evident from the evaluation. I hope to see more improvements in this scanner, or in general, in the field of open-source software security.
%